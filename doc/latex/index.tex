This homework is using C++ for Programming Language and without S\-T\-L. It has classes for representing \hyperlink{class_date}{Date} and \hyperlink{class_time}{Time} and both of them. It also has class \hyperlink{class_queue}{Queue} and \hyperlink{class_teller}{Teller} for representing Banks'\hyperlink{class_queue}{Queue}. Class \hyperlink{class_event}{Event} is a controlling class for user's input. There is a main file that use all of them named \hyperlink{main_8cpp}{main.\-cpp}.

{\bfseries Problem Description \-:}\par
 Pada sebuah bank, ada N teller, di depan masing-\/masing teller ada antrian. Ada N buah teller dinomori T(0) s.\-d T(N-\/1). Q(i) adalah antrian di depan teller T(i).\par


T(i) statusnya 0\-: menganggur/idle atau 1\-: sedang melayani\par


Program utama akan memroses sederetan “event” yang diberikan sebagai input, dan dipastikan eventnya terurut waktu dan membesar (mencerminkan kejadian sesuai dengan berjalannya waktu). Jika ada event yang waktunya bersamaan, anda harus memroses sesuai urutan input. Sebuah event adalah type yang terdiri dari tiga komponen yaitu $<$T\-: \hyperlink{class_date_time}{Date\-Time}; Kode\-:char; i\-:integer$>$ yang dijelaskan sebagai berikut \-: \par
 
\begin{DoxyEnumerate}
\item T adalah type \hyperlink{class_date_time}{Date\-Time}, dengan \hyperlink{class_date}{Date} dan \hyperlink{class_time}{Time} yang harus dibuat sendiri, dengan method yang hanya diperlukan untuk persoalan. Format Input \hyperlink{class_date_time}{Date\-Time} \-: D\-D-\/\-M\-M-\/\-Y\-Y\-Y\-Y;H\-H\-:\-M\-M\-:S\-S 
\item Kode adalah sebuah karakter yang bernilai ‘\-A’ atau ‘\-D’. A = Arrival (kedatangan pelanggan) dan D = Departure, seorang pelanggan selesai dilayani sehingga harus dihapus dari \hyperlink{class_queue}{Queue}. 
\item i adalah nomor I\-D pelanggan (di-\/generate secara otomatis oleh program anda terurut mulai dari 1 pada saat Arrival). 
\end{DoxyEnumerate}

Program anda akan memproses deretan event yang diberikan sesuai urutan pembacaan secara sekuensial, dan akan berhenti jika T $>$= Tmax, yang merupakan jam tutup teller. Jika Tmax tercapai, program harus berhenti menangani deretan event, dan akan melakukan penghapusan ke semua pelanggan yang sedang tersisa dengan “merata” artinya ulangi hapus satu per satu dari Q\mbox{[}1\mbox{]}, Q\mbox{[}2\mbox{]},..Q\mbox{[}N\mbox{]}. Merata artinya bukan menghabiskan sebuah \hyperlink{class_queue}{Queue} sampai kosong sekaligus, tapi digilir penghapusannya.

{\bfseries Jockeying}\par
 

Fenomena “jockeying” dalam sebuah antrian adalah terjadinya seorang pelanggan pindah ke antrian lain, karena sesuatu sebab. Yang paling sering adalah karena melihat bahwa \hyperlink{class_queue}{Queue} di dekatnya lebih pendek. Padahal, belum tentu kalau antrian lebih pendek akan lebih cepat dilayani sebab tergantung kepada lamanya pelayanan pelanggan. Fenomena jockeying dapat menyebabkan pelanggan hanya berpindah-\/pindah antrian tapi malahan tidak terlayani. Pada Tugas Besar ini, anda akan membuat sebuah algoritma simulasi jockeying ke antrian ke-\/j ketika terjadi departure di sebuah antrian ke-\/i karena banyaknya pelanggan yang mengantri di j menjadi lebih kecil dari banyaknya yang mengantri di antrian ke-\/i.

Spesifikasi “jockeying” adalah sebagai berikut \-:\par
 int Jockeying(int i\-Origin)\par
 /$\ast$ i\-Origin = nomor \hyperlink{class_queue}{Queue} asal\par
 Fungsi Jockeying menghasilkan j yaitu nomor \hyperlink{class_queue}{Queue} tujuan (jika terjadi jockeying),atau -\/1 (jika tidak terjadi jockeying)\par
 Syarat terjadinya jockeying \-: Ada queue lain yang lebih pendek dengan selisih lebih dari 2 elemen\par
 Mensimulasikan pelanggan yang berpindah dari Q\mbox{[}i\-Origin\mbox{]} ke Q\mbox{[}j\mbox{]} (jika ada)., dengan j != i\-Origin dan Nb\-Elmt(j) paling minimum. \par
 Jika terdapat lebih dari satu Q\mbox{[}j\mbox{]} dengan Nb\-Elmt(j) minimum, pilih j yang paling dekat dengan i\-Origin dan j yang memiliki indeks lebih kecil. \par
 